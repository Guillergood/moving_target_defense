\chapter{Glosario de términos}

\textbf{Alpine Linux}: Es una distribución Linux basada en BusyBox y la librería \texttt{musl}. Debido a su pequeño tamaño, es muy utilizado en plataformas de contenedores ya que proporciona tiempos de arranque muy rápidos.
\bigskip

\textbf{API (application programming interface)}: Son un conjunto de funciones que funcionan como una capa de abstracción para poder interactuar con un determinado software.
\bigskip

\textbf{Apache Benchmark}: Herramienta para realizar pruebas de carga de servidores web desarrollada por la Fundación Apache.
\bigskip

\textbf{Auditoría de seguridad}: Estudio que comprende el análisis de sistemas para identificar, enumerar y posteriormente describir las diversas vulnerabilidades que pudieran presentarse en una revisión exhaustiva de las estaciones de trabajo, redes de comunicaciones o servidores.
\bigskip

\textbf{Backend}: Es el motor de una aplicación, se encarga de realizar las funciones en segundo plano que se encargan de que la aplicación funcione.
\bigskip

\textbf{Balanceo de carga}: Técnica de configuración de servidores que permite que la carga de trabajo total se reparte entre varios de ellos para que no disminuya el rendimiento general de la infraestructura.
\bigskip

\textbf{Caddy}: Es un servidor web de código abierto escrito en Go. Utiliza la funcionalidad HTTP de la biblioteca estándar de Go.
\bigskip

\textbf{Clickjacking}: es una técnica maliciosa que consiste en engañar al usuario para que haga clic en algo diferente de lo que el usuario cree, revelando así información confidencial o permitiendo que otros tomen el control de su ordenador.
\bigskip

\textbf{CVSS}: El `Common Vulnerability Scoring System' es un formato abierto de métricas para la comunicación de vulnerabilidades.
\bigskip

\textbf{DEAP}: Es un \textit{framework} de código abierto abierto para la realización de algoritmos genéticos. Está desarrollado en Python.
\bigskip

\textbf{Exploit}: Es un código que se utiliza con fin de aprovechar una vulnerabilidad en un determinado software consiguiendo un comportamiento no deseado en el mismo.
\bigskip

\textbf{Expresión regular}: Secuencia de caracteres que forma un patrón de búsqueda, principalmente utilizada para la búsqueda de patrones de cadenas de caracteres u operaciones de sustitución.
\bigskip

\textbf{Frontend}: Es la interfaz de la aplicación, es la parte de la aplicación que el usuario utiliza para comunicarse con la misma.
\bigskip

\textbf{Fundación Mozilla}: Organización sin ánimo de lucro que desarrolla el navegador Firefox así como otros programas licenciados software libre.
\bigskip

\textbf{Hash}: Algoritmo que a partir de una entrada genera una salida alfanumérica que representa un resumen de dicha entrada. Un simple cambio en los datos de la entrada generará un resumen totalmente diferente.
\bigskip

\textbf{HTML (HyperText Markup Language)}: Lenguaje de marcado que se utiliza para la realización de páginas web.
\bigskip

\textbf{Iptables}: Es un poderoso \textit{firewall} integrado en el núcleo de \texttt{Linux} y que forma parte del proyecto \texttt{netfilter}.
\bigskip

\textbf{JavaScript}: Lenguaje de programación orientado a objetos interpretado que se utiliza principalmente para cargar programas desde el lado del cliente en los navegadores web.
\bigskip

\textbf{JSON (JavaScript Object Notation)}: Formato de texto plano usado para el intercambio de información, independientemente del lenguaje de programación. Es una simplificación del formato XML siendo mucho mas legible por humanos.
\bigskip

\textbf{LaTeX}: Sistema de composición de documentos que permite crear textos en diferentes formatos (artículos, cartas, libros, informes...) obteniendo una alta calidad en los documentos generados.
\bigskip

\textbf{MIME}: Las `Multipurpose Internet Mail Extensions', por sus siglas en inglés, son una especificación dirigida al intercambio de ficheros de diferentes formatos a través de Internet. Se usa en mayor medida para reconocer el tipo de dichos ficheros.
\bigskip

\textbf{Musl}: es una implementación de la librería estándar de C de pequeño tamaño destinada a sistemas operativos embebidos basados en el kernel de Linux, está publicada bajo la licencia MIT.
\bigskip

\textbf{OWASP ZAP}: `El Zed Attack Proxy' es una herramienta para realizar análisis de seguridad en aplicaciones web desarrollada por OWASP. Es uno de los proyectos de OWASP más activos.
\bigskip

\textbf{OWASP}: El `Open Web Application Security Project', por sus siglas en inglés, es una organización cuya misión es mejorar la seguridad del software. Ha desarrollado diversas herramientas así como estándares de comunicación de vulnerabilidades. Además cuenta con un popular \textit{ranking} anual de las 10 vulnerabilidades que consideran críticas en aplicaciones web de mayor y ofrece recomendaciones sobre como evitarlas.
\bigskip

\textbf{Prueba unitaria}: Es una técnica que sirve para comprobar de forma atómica el funcionamiento del codigo de un sistema software. Su uso está aconsejado, \textit{TDD} se basa en este tipo de pruebas.
\bigskip

\textbf{Ransomware}: es un tipo de \textit{malware} que restringe el acceso a los datos almacenados en un sistema informático, comúnmente mediante técnicas de cifrado, y exige que se pague un rescate al creador para restaurar dicho acceso.
\bigskip

\textbf{RSA}: Es un sistema de criptografía basado en clave pública y privada. Es usado para securizar la transferencia de datos.
\bigskip

\textbf{Sniffer}: También conocido como analizador de paquetes, es un programa informático que puede interceptar y registrar el tráfico que pasa por una red.
\bigskip

\textbf{Software libre}: Software cuya licencia permite que este sea usado, copiado, modificado y distribuido libremente según el tipo de licencia que adopte.
\bigskip

\textbf{SSH (Secure SHell)}: Protocolo que permite conectarse a máquinas remotas mediante conexiones seguras de red.
\bigskip

\textbf{SSL (Secure Sockets Layer)}: Serie de protocolos criptográficos que proporcionan comunicaciones seguras por una red.
\bigskip

\textbf{StatusCake}: Es un servicio web que permite monitorizar la disponibilidad de servicios online.
\bigskip

\textbf{STIG}: Las `Security Technical Implementation Guides' son unas directrices de ciberseguridad para estandarizar los protocolos de seguridad dentro de redes de ordenadores para mejorar la seguridad general. Estas guías, cuando se implementan, mejoran la seguridad del software, hardware, y arquitecturas físicas para reducir la exposición a vulnerabilidades.
\bigskip

\textbf{TDD}: El `Test Driven Development', por sus siglas en inglés, es una técnica de desarrollo en las que primero se escriben pruebas unitarias del código y luego se va desarrollando el código de forma incremental hasta que pasa las diferentes pruebas unitarias.
\bigskip

\textbf{URL (Uniform Resource Locator)}: nombre y con un formato estándar que permite acceder a un recurso de forma inequívoca.
\bigskip

\textbf{WireShark}: Es un analizador de red utilizado para auditar redes de comunicaciones.
\bigskip

\textbf{YML}: `YAML Ain't Markup Language' por sus siglas en inglés es un formato de archivo para almacenar datos serializados de forma legible para humanos. Es compatible con el formato JSON con la ventaja de que ademas de legible es fácilmente editable.
