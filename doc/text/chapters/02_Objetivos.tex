\chapter{Objetivos}

El objetivo máximo de este proyecto es la elaboración de un sistema software que mediante el uso de algoritmos genéticos pueda optimizar la configuración de un determinado servicio y que aplicando dicha configuración de forma regular la seguridad del servicio se vea incrementada al entorpecer la recopilación de información realizada por un posible atacante.

\bigskip
Dicho objetivo se descompone los siguientes objetivos principales:

\begin{itemize}
  \item \textbf{OBJ-1.} Desarrollar una herramienta para prevenir ataques informáticos.
  \item \textbf{OBJ-2.} Analizar distintos servicios web candidatos a ser optimizados y securizados.
  \item \textbf{OBJ-3.} Analizar las posibilidades de configuración de los servicios anteriores.
  \item \textbf{OBJ-4.} Determinar la ventaja que nos brindan los algoritmos genéticos para la resolución de problemas.
\end{itemize}

Además como objetivos secundarios tendremos:

\begin{itemize}
  \item \textbf{OBJ-5.} Cuantificar la seguridad de una determinada configuración.
  \item \textbf{OBJ-6.} Cuantificar la optimización de una determinada configuración.
\end{itemize}


\section{Alcance de los objetivos}
El fin inmediato de este informe es desarrollar una herramienta que permita a los administradores de sistemas securizar servidores de una forma sencilla.
Además todo el código así como la documentación resultante se liberará con una licencia libre para que cualquiera pueda hacer uso de las conclusiones y los datos extraídos del análisis.

\section{Interdependencia de los objetivos}

Todos los objetivos son interdependientes entre sí, pero el primer objetivo (\textbf{OBJ-1}) es el principal motivador de este proyecto, por lo que aún sin representar el desarrollo de ningún trabajo en concreto es el que va a escudar y avalar el desarrollo de los otros. En aspectos más relacionados con la realización del proyecto, el tercer objetivo (\textbf{OBJ-3}) es el que nos brindará el sistema sobre la que trabajar, ya que sienta la base sobre la que aplicar el cuarto objetivo (\textbf{OBJ-4}). El resto de objetivos secundarios, al no tener un carácter urgente serán resueltos en base a la disponibilidad del tiempo necesario para su realización.

\section{Conocimientos y herramientas utilizadas}

\bigskip
Destacar en los aspectos formativos previos más utilizados para el desarrollo del proyecto los conocimientos adquiridos en las asignaturas ``Cloud Computing'' para el análisis y configuración de los diferentes servicios de red así como todo lo referente a virtualización de sistemas, ``Inteligencia Computacional'' para el desarrollo del algoritmo genético y ``Planificación y Gestión de Proyectos Informáticos'' para definir los requisitos y el planteamiento inicial del proyecto, siendo todas ellas del \textbf{\master}. También destacar las asignaturas del \textbf{Grado en Ingeniería Informática} ``Seguridad en Sistemas Operativos'' para la parte de seguridad y ``Servidores Web de Altas Prestaciones'' para la realización de pruebas desde el punto de vista de disponibilidad y carga de trabajo.

\bigskip
Para la realización de cada una de las partes se han usado multitud de herramientas específicas tales como \texttt{LaTeX}, \texttt{Zotero}, \texttt{Docker} \texttt{Git} y \texttt{OWASP ZAP} entre otras.

