\chapter{Antecedentes}

En este capítulo vamos analizar el estado de arte actual y las tecnologías candidatas a utilizarse en este proyecto en base a los objetivos que presentamos en el capítulo anterior.

\bigskip
Como ya vimos en la introducción, diversos estudios aseguran que es posible incrementar la seguridad de una configuración en base al uso de algoritmos genéticos, pero la mayoría son demostraciones teóricas sin ninguna implementación real cuantificable  \cite{john_evolutionary_2014} \cite{romero_sistema_2017} \cite{buji_genetic_2017}. 

\section{Herramientas existentes}
A pesar de las multiples aproximaciones teóricas \cite{schlenker_deceiving_2018} \cite{champagne_genetic_2018}, actualmente no existe ninguna herramienta, ni ningún ejemplo liberado de forma pública que permita comprobar y generar configuraciones e ir evolucionando las mismas para mejorar tanto su seguridad como su diversidad por lo que se ha optado por la realización de una herramienta capaz de realizar esto.

\bigskip
Para el análisis y cuantificación de vulnerabilidades si existe herramientas, además algunas de ellas liberadas con licencias abiertas. Hemos optado por OWASP ZAP ya que es uno de las herramientas de análisis de vulnerabilidades más utilizadas y además está liberada bajo la licencia GPLv3 \cite{free_software_foundation_gnu_2007}.

\section {Protocolos y/o servidores de red}

En esta sección analizaremos algunos de los protocolos de red más conocidos así como algunas de sus implementaciones. Ateniéndonos a la filosofía abierta de este proyecto nos limitaremos a las implementaciones libres de los mismos, además, diversas publicaciones demuestran que el software de código abierto es mas seguro que el software cerrado \cite{walia_comparative_2006} \cite{mansfield-devine_open_2008} \cite{clark_is_2009}.

\subsection {SMB - Server Message Block}

Es un protocolo de red desarrollado por IBM a principios de la década de los 90 y adoptado por Microsoft a partir de 1992. Este protocolo permite compartir archivos e impresoras en red.

\bigskip
Debido a la cantidad de sistemas compatibles es uno de los protocolos más utilizados para compartir ficheros en redes empresariales, esto hace que sea uno de los objetivos principales de muchos ciberataques.

\bigskip
Por poner un ejemplo, en mayo de 2017 hubo un ciberataque masivo causado por el ransomware WannaCry \cite{sarabia_mayor_2017}, dicho software hacía uso de un \textit{exploit} conocido como EternalBlue que conseguía penetrar en sistemas que todavía hacían uso de los protocolos SMB1 y SMB2 obligando a MicroSoft a publicar parches incluso para sistemas operativos que habían terminado su ciclo de vida (Windows XP y Windows Vista).


\subsubsection {Implementaciones libres}

\begin{table}[H]
\begin{tabular}{|l|l|}
\hline
Nombre                   & Samba                        \\ \hline
Licencia                 & GPLv3                        \\ \hline
Año de lanzamiento       & 1992                         \\ \hline
Lenguaje de programación & C++, Python y C              \\ \hline
Sitio Web                & \url{https://www.samba.org} 	\\ \hline
\end{tabular}
\caption{Ficha técnica Samba}
\end{table}

Samba es una implementación libre del protocolo usado para compartir archivos de Microsoft desarrollado originalmente para Unix por Andrew Tridgell utilizando técnicas de ingeniería inversa para averiguar el funcionamiento del protocolo. Aunque sigue siendo un protocolo propietario a partir de la versión 2.0 (2006) Microsoft comenzó a publicar las especificaciones del protocolo SMB para permitir la interoperatibilidad entre diferentes sistemas operativos.


\subsection {NTP - Network Time Protocol}

Network Time Protocol (NTP) es un protocolo utilizado para sincronizar los relojes de diversos sistemas informáticos.

\subsubsection {Implementaciones libres}

\begin{table}[H]
\begin{tabular}{|l|l|}
\hline
Nombre                   & OpenNTPD                       \\ \hline
Licencia                 & ISC                            \\ \hline
Año de lanzamiento       & 2004                           \\ \hline
Lenguaje de programación & C                              \\ \hline
Sitio Web                & \url{http://www.openntpd.org}  \\ \hline
\end{tabular}
\caption{Ficha técnica OpenNTPD}
\end{table}

OpenNTPD es una implementación del protocolo NTP para sincronizar el reloj del sistema contra servidores NTP remotos. También puede actuar como un servidor NTP para clientes compatibles con NTP. OpenNTPD está desarrollado principalmente por Henning Brauer como parte del proyecto OpenBSD.

\bigskip
La motivación para desarrollar OpenNTPD fue una combinación de problemas con las implementación NTP existentes como pueden ser una configuración difícil, código complejo y difícil de auditar así como licencias incompatibles con la licencia BSD.


\subsection {VPN - Virtual Private Network}

Una red privada virtual (VPN) es una tecnología que permite crear una red segura de acceso local (LAN) sobre una red pública como puede ser Internet.

\bigskip
El protocolo más utilizado es IPSEC, pero también existen protocolos como pueden ser PPTP y L2TP. Cada uno con sus ventajas y desventajas en cuanto a seguridad, facilidad, mantenimiento y tipos de clientes soportados.

\subsubsection {Implementaciones libres}

\begin{table}[H]
\begin{tabular}{|l|l|}
\hline
Nombre                   & strongSwan                       \\ \hline
Licencia                 & GPLv2                            \\ \hline
Año de lanzamiento       & 2005                           \\ \hline
Lenguaje de programación & C                              \\ \hline
Sitio Web                & \url{http://www.strongswan.org} \\ \hline
\end{tabular}
\caption{Ficha técnica strongSwan}
\end{table}

\begin{table}[H]
\begin{tabular}{|l|l|}
\hline
Nombre                   & Libreswan                       \\ \hline
Licencia                 & GPL                            \\ \hline
Año de lanzamiento       & 2013                           \\ \hline
Lenguaje de programación & C                              \\ \hline
Sitio Web                & \url{http://libreswan.org} \\ \hline
\end{tabular}
\caption{Ficha técnica Libreswan}
\end{table}


\subsection {SSH - Secure Shell}

El protocolo SSH (Secure SHel) sirve para abrir un intérprete de comandos en un servidor remoto.

\bigskip
Además de dicho intérprete de comandos, SSH nos permite copiar datos de forma segura (scp), gestionar claves de tipo publica/privada para no requerir el uso de contraseñas y pasar los datos de cualquier otra aplicación por un canal seguro a través de túneles.

\subsubsection {Implementaciones libres}

\begin{table}[H]
\begin{tabular}{|l|l|}
\hline
Nombre                   & Dropbear                       \\ \hline
Licencia                 & MIT                            \\ \hline
Año de lanzamiento       & 2003                           \\ \hline
Lenguaje de programación & C                              \\ \hline
Sitio Web                & \url{http://matt.ucc.asn.au/dropbear/dropbear.html} \\ \hline
\end{tabular}
\caption{Ficha técnica Dropbear}
\end{table}

Dropbear es un servidor SSH desarrollado por Matt Johnston. Está diseñado para entornos con pocos recursos como pueden ser sistemas embebidos por lo que es ampliamente utilizado en routers basados en Linux como puede ser OpenWRT.

\begin{table}[H]
\begin{tabular}{|l|l|}
\hline
Nombre                   & OpenSSH                       \\ \hline
Licencia                 & BSD                            \\ \hline
Año de lanzamiento       & 1999                           \\ \hline
Lenguaje de programación & C                              \\ \hline
Sitio Web                & \url{http://www.openssh.com} \\ \hline
\end{tabular}
\caption{Ficha técnica Libreswan}
\end{table}

OpenSSH es una implementación libre del protocolo SSH desarrollado por Theo de Raadt, fundador también del proyecto OpenBSD.

\subsection {FTP - File Transfer Protocol}

FTP (File Transfer Protocol) es uno de los primero protocolos diseñado para la transferencia de archivos entre sistemas utilizando una arquitectura cliente/servidor. A pesar de sus problemas de seguridad \cite{todd_why_2000} es uno de los protocolos más utilizados para transferir archivos. Dichos problemas se deben a que el todo el intercambio de información, desde la autenticación del usuario hasta la transferencia de, archivo, se realiza en texto plano sin ningún tipo de cifrado. Hoy en día se aconseja el uso de otros métodos mas eficaces como los programas rsync o scp basados a su vez en SSH. El propio repositorio de código de \texttt{vsftpd} fue comprometido en 2011 \cite{hkcert_security_bulletin_sa11070501_2011}.

\subsubsection {Implementaciones libres}

\begin{table}[H]
\begin{tabular}{|l|l|}
\hline
Nombre                   & vsftpd                       \\ \hline
Licencia                 & GPLv2                        \\ \hline
Año de lanzamiento       & 2001                         \\ \hline
Lenguaje de programación & C                            \\ \hline
Sitio Web                & \url{https://security.appspot.com/vsftpd.html} \\ \hline
\end{tabular}
\caption{Ficha técnica vsftp}
\end{table}

vsftpd es un servidor FTP con licencia GPL para sistemas UNIX.

\subsection {HTTP - Hypertext Transfer Protocol}

El protocolo de transferencia de hipertexto, HTTP por sus siglas en inglés, es un protocolo de red. Se utiliza para enviar y recibir páginas web en Internet. Fue desarrollado por Tim Berners-Lee y está coordinado por el \textbf{World Wide Web Consortium} (W3C).

\subsubsection {Implementaciones libres}

\begin{table}[H]
\begin{tabular}{|l|l|}
\hline
Nombre                   & Apache                       \\ \hline
Licencia                 & Apache 2.0                        \\ \hline
Año de lanzamiento       & 1995                         \\ \hline
Lenguaje de programación & C                            \\ \hline
Sitio Web                & \url{https://httpd.apache.org} \\ \hline
\end{tabular}
\caption{Ficha técnica Apache}
\end{table}

El servidor HTTP Apache es un servidor web HTTP de código abierto desarrollado y mantenido por la Fundación Apache, siendo uno de los servidores HTTP más utilizados del mudo.

\begin{table}[H]
\begin{tabular}{|l|l|}
\hline
Nombre                   & NGINX                       \\ \hline
Licencia                 & BSD                        \\ \hline
Año de lanzamiento       & 2004                        \\ \hline
Lenguaje de programación & C                            \\ \hline
Sitio Web                & \url{http://nginx.org} \\ \hline
\end{tabular}
\caption{Ficha técnica NGINX}
\end{table}

NGINX es un servidor web de alto rendimiento inicialmente desarrollado por Igor Sysoev siendo uno de los mas utilizados en la actualidad.

\section {Algoritmos genéticos}

Un algoritmo genético es un algoritmo de optimización que imita el proceso de selección natural descrito por Charles Darwin en su teoría de la evolución y nos puede ayudar a resolver problemas de optimización y búsqueda imitando procesos biológicos naturales, como pueden ser la mutación, la selección y el cruzamiento \cite{batista_algoritmos_2009}.

\bigskip
Los algoritmos genéticos son heurísticas de búsqueda que se utilizan a menudo para encontrar soluciones complejas y no obvias a la optimización algorítmica y a los problemas de búsqueda \cite{orcero_inteligencia_2002}.

\bigskip
Los algoritmos genéticos se basan en un conjunto de individuos de una población natural, codificando la información de cada solución en una cadena llamada cromosoma. Los símbolos que forman la cadena son llamados genes.

\bigskip
Los cromosomas evolucionan a través de iteraciones, llamadas generaciones. En cada generación, los cromosomas son evaluados usando alguna medida de aptitud. Las siguientes generaciones (nuevos cromosomas), son generadas aplicando los operadores genéticos de selección, mutación y cruzamiento repetidamente.


