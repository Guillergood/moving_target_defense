\chapter{Anexos}

\section{Codigo Fuente}

\subsection{Algoritmo Genético}
\lstinputlisting[language=python,label={genetic.py},caption={genetic.py}]{../../code/genetic.py}

\subsection{Generador configuración NGINX}
\lstinputlisting[language=python,label={generate_nginx_config.py},caption={generate\_nginx\_config.py}]{../../code/generate_nginx_config.py}

\subsection{Analizador con ZAP}
\lstinputlisting[language=python,label={zap.py},caption={zap.py}]{../../code/zap.py}

\subsection{Algoritmo de fitness}
\lstinputlisting[language=python,label={fitness.py},caption={fitness.py}]{../../code/fitness.py}

\newpage
\subsection{Dockerfile}
\lstinputlisting[language=docker,label={Dockerfile},caption={Dockerfile}]{../../code/Dockerfile}

\subsection{docker-compose}
\lstinputlisting[language=docker-compose,label={docker-compose.yml},caption={docker-compose.yml}]{../../code/docker-compose.yml}

\subsection{Pruebas unitarias}
\lstinputlisting[language=python,label={test_genetic.py},caption={test\_genetic.py}]{../../code/test/test_genetic.py}
\lstinputlisting[language=python,label={test_generate_nginx_config.py},caption={test\_generate\_nginx\_config.py}]{../../code/test/test_generate_nginx_config.py}
\lstinputlisting[language=python,label={test_fitness.py},caption={test\_fitness.py}]{../../code/test/test_fitness.py}

\section{Salidas de programas}

\begin{lstlisting}[language=json,label={lst:owas_zap_welcome_message_alerts},caption={Alerts showed with NGINX default configuration}]
[{'alert': 'Web Browser XSS Protection Not Enabled',
  'attack': '',
  'confidence': 'Medium',
  'cweid': '933',
  'description': 'Web Browser XSS Protection is not enabled, or is disabled by '
                 "the configuration of the 'X-XSS-Protection' HTTP response "
                 'header on the web server',
  'evidence': '',
  'id': '0',
  'messageId': '1',
  'method': 'GET',
  'name': 'Web Browser XSS Protection Not Enabled',
  'other': 'The X-XSS-Protection HTTP response header allows the web server to '
           "enable or disable the web browser's XSS protection mechanism. The "
           'following values would attempt to enable it: \n'
           'X-XSS-Protection: 1; mode=block\n'
           'X-XSS-Protection: 1; report=http://www.example.com/xss\n'
           'The following values would disable it:\n'
           'X-XSS-Protection: 0\n'
           'The X-XSS-Protection HTTP response header is currently supported '
           'on Internet Explorer, Chrome and Safari (WebKit).\n'
           'Note that this alert is only raised if the response body could '
           'potentially contain an XSS payload (with a text-based content '
           'type, with a non-zero length).',
  'param': 'X-XSS-Protection',
  'pluginId': '10016',
  'reference': 'https://www.owasp.org/index.php/XSS_(Cross_Site_Scripting)_Prevention_Cheat_Sheet\n'
               'https://www.veracode.com/blog/2014/03/guidelines-for-setting-security-headers/',
  'risk': 'Low',
  'solution': "Ensure that the web browser's XSS filter is enabled, by setting "
              "the X-XSS-Protection HTTP response header to '1'.",
  'sourceid': '3',
  'url': 'http://www.example.com',
  'wascid': '14'},
 {'alert': 'X-Content-Type-Options Header Missing',
  'attack': '',
  'confidence': 'Medium',
  'cweid': '16',
  'description': 'The Anti-MIME-Sniffing header X-Content-Type-Options was not '
                 "set to 'nosniff'. This allows older versions of Internet "
                 'Explorer and Chrome to perform MIME-sniffing on the response '
                 'body, potentially causing the response body to be '
                 'interpreted and displayed as a content type other than the '
                 'declared content type. Current (early 2014) and legacy '
                 'versions of Firefox will use the declared content type (if '
                 'one is set), rather than performing MIME-sniffing.',
  'evidence': '',
  'id': '1',
  'messageId': '1',
  'method': 'GET',
  'name': 'X-Content-Type-Options Header Missing',
  'other': 'This issue still applies to error type pages (401, 403, 500, etc) '
           'as those pages are often still affected by injection issues, in '
           'which case there is still concern for browsers sniffing pages away '
           'from their actual content type.\n'
           'At "High" threshold this scanner will not alert on client or '
           'server error responses.',
  'param': 'X-Content-Type-Options',
  'pluginId': '10021',
  'reference': 'http://msdn.microsoft.com/en-us/library/ie/gg622941%28v=vs.85%29.aspx\n'
               'https://www.owasp.org/index.php/List_of_useful_HTTP_headers',
  'risk': 'Low',
  'solution': 'Ensure that the application/web server sets the Content-Type '
              'header appropriately, and that it sets the '
              "X-Content-Type-Options header to 'nosniff' for all web pages.\n"
              'If possible, ensure that the end user uses a '
              'standards-compliant and modern web browser that does not '
              'perform MIME-sniffing at all, or that can be directed by the '
              'web application/web server to not perform MIME-sniffing.',
  'sourceid': '3',
  'url': 'http://www.example.com',
  'wascid': '15'},
 {'alert': 'X-Frame-Options Header Not Set',
  'attack': '',
  'confidence': 'Medium',
  'cweid': '16',
  'description': 'X-Frame-Options header is not included in the HTTP response '
                 "to protect against 'ClickJacking' attacks.",
  'evidence': '',
  'id': '2',
  'messageId': '1',
  'method': 'GET',
  'name': 'X-Frame-Options Header Not Set',
  'other': '',
  'param': 'X-Frame-Options',
  'pluginId': '10020',
  'reference': 'http://blogs.msdn.com/b/ieinternals/archive/2010/03/30/combating-clickjacking-with-x-frame-options.aspx',
  'risk': 'Medium',
  'solution': 'Most modern Web browsers support the X-Frame-Options HTTP '
              "header. Ensure it's set on all web pages returned by your site "
              '(if you expect the page to be framed only by pages on your '
              "server (e.g. it's part of a FRAMESET) then you'll want to use "
              'SAMEORIGIN, otherwise if you never expect the page to be '
              'framed, you should use DENY. ALLOW-FROM allows specific '
              'websites to frame the web page in supported web browsers).',
  'sourceid': '3',
  'url': 'http://www.example.com',
  'wascid': '15'},
 {'alert': 'Web Browser XSS Protection Not Enabled',
  'attack': '',
  'confidence': 'Medium',
  'cweid': '933',
  'description': 'Web Browser XSS Protection is not enabled, or is disabled by '
                 "the configuration of the 'X-XSS-Protection' HTTP response "
                 'header on the web server',
  'evidence': '',
  'id': '6',
  'messageId': '7',
  'method': 'GET',
  'name': 'Web Browser XSS Protection Not Enabled',
  'other': 'The X-XSS-Protection HTTP response header allows the web server to '
           "enable or disable the web browser's XSS protection mechanism. The "
           'following values would attempt to enable it: \n'
           'X-XSS-Protection: 1; mode=block\n'
           'X-XSS-Protection: 1; report=http://www.example.com/xss\n'
           'The following values would disable it:\n'
           'X-XSS-Protection: 0\n'
           'The X-XSS-Protection HTTP response header is currently supported '
           'on Internet Explorer, Chrome and Safari (WebKit).\n'
           'Note that this alert is only raised if the response body could '
           'potentially contain an XSS payload (with a text-based content '
           'type, with a non-zero length).',
  'param': 'X-XSS-Protection',
  'pluginId': '10016',
  'reference': 'https://www.owasp.org/index.php/XSS_(Cross_Site_Scripting)_Prevention_Cheat_Sheet\n'
               'https://www.veracode.com/blog/2014/03/guidelines-for-setting-security-headers/',
  'risk': 'Low',
  'solution': "Ensure that the web browser's XSS filter is enabled, by setting "
              "the X-XSS-Protection HTTP response header to '1'.",
  'sourceid': '3',
  'url': 'http://www.example.com/robots.txt',
  'wascid': '14'},
 {'alert': 'Web Browser XSS Protection Not Enabled',
  'attack': '',
  'confidence': 'Medium',
  'cweid': '933',
  'description': 'Web Browser XSS Protection is not enabled, or is disabled by '
                 "the configuration of the 'X-XSS-Protection' HTTP response "
                 'header on the web server',
  'evidence': '',
  'id': '7',
  'messageId': '8',
  'method': 'GET',
  'name': 'Web Browser XSS Protection Not Enabled',
  'other': 'The X-XSS-Protection HTTP response header allows the web server to '
           "enable or disable the web browser's XSS protection mechanism. The "
           'following values would attempt to enable it: \n'
           'X-XSS-Protection: 1; mode=block\n'
           'X-XSS-Protection: 1; report=http://www.example.com/xss\n'
           'The following values would disable it:\n'
           'X-XSS-Protection: 0\n'
           'The X-XSS-Protection HTTP response header is currently supported '
           'on Internet Explorer, Chrome and Safari (WebKit).\n'
           'Note that this alert is only raised if the response body could '
           'potentially contain an XSS payload (with a text-based content '
           'type, with a non-zero length).',
  'param': 'X-XSS-Protection',
  'pluginId': '10016',
  'reference': 'https://www.owasp.org/index.php/XSS_(Cross_Site_Scripting)_Prevention_Cheat_Sheet\n'
               'https://www.veracode.com/blog/2014/03/guidelines-for-setting-security-headers/',
  'risk': 'Low',
  'solution': "Ensure that the web browser's XSS filter is enabled, by setting "
              "the X-XSS-Protection HTTP response header to '1'.",
  'sourceid': '3',
  'url': 'http://www.example.com/sitemap.xml',
  'wascid': '14'}]
\end{lstlisting}