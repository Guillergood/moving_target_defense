\chapter{Resultados}

En este capítulo pasamos a detallar los resultados de todos y cada uno de los test que realizamos en base a la metodología que definimos en el capítulo anterior.

\section{Análisis inicial del sistema}

Con nuestro generador de código podemos generar configuraciones aleatorias de NGINX como la que podemos ver en el listado ver \ref{lst:ngin_config_random}. Aunque como ya hemos comentado previamente esa configuración puede ser errónea o menos segura que otra configuración también generada aleatoriamente. En este caso concreto la configuración es claramente errónea porque ha seleccionado elementos que harían que NGINX no pudiera ni siquiera inicializarse.

\bigskip
Los cromosomas representan una combinación de configuraciones que tiene un tamaño construido de acuerdo a los parámetros de las configuraciones donde un alelo o más representan un parámetro. En esta investigación, el tamaño de los cromosomas es de 13 y el total de individuos en una generación es de 20, lo que es suficiente para mantener la diversidad de los parámetros de acuerdo con el número de parámetros. El número de soluciones en una generación ha sido determinado después de muchos experimentos del algoritmo. Es posible aumentar el número de cromosomas en una generación, lo que podría aumentar la diversidad, pero eso costaría más tiempo para las pruebas.

\bigskip
Para validar que nuestros datos son correctos se han introducido algunos errores en la configuración de forma intencional. Las cabeceras `X-Frame-Options' y `X-Powered-By' contienen unos valores incorrectos al final, por lo que el gen 10 nunca podrá tener un valor de 3, asimismo el gen 11 nunca podrá tener un valor de 5. Con esto nos aseguramos que efectivamente las configuraciones incorrectas nunca serán dadas por válidas.

\bigskip
A continuación pasamos a ejecutar nuestras pruebas utilizando diferentes valores y utilizando dos funciones de cruce distintas. Una funciona de cruzamiento lo hará en un único punto aleatorio y el otro lo hará en dos.

\bigskip
Se han generado los individuos de forma aleatoria y se ha ejecutado el algoritmo genético durante las generaciones indicadas, podemos ver que hay configuraciones incorrectas (en rojo).

\bigskip
Tras cada ejecución obtenemos una configuración de NGINX que podríamos aplicar en nuestro servidor.

\section{Cruzamiento en un único punto}

\subsection{Población de 10 individuos durante 2 generaciones}
Resultados para una población de 10 individuos durante 2 generaciones:
Población inicial:
\begin{table}[H]
\begin{tabular}{|l|l|l|l|l|l|l|l|l|l|l|l|l|}
\hline
\textbf{1} & \textbf{2} & \textbf{3} & \textbf{4} & \textbf{5} & \textbf{6} & \textbf{7} & \textbf{8} & \textbf{9} & \textbf{10} & \textbf{11} & \textbf{12} & \textbf{13} \\ \hline
1884  &  75  &  1  &  1  &  0  &  805  &  1491  &  1  &  1  &  0  &  {\color[HTML]{FE0000}4}  &  1  &  1 \\ \hline
666  &  17  &  0  &  1  &  1  &  659  &  1251  &  0  &  1  &  0  &  {\color[HTML]{FE0000}5}  &  1  &  1 \\ \hline
1515  &  19  &  1  &  1  &  0  &  1104  &  1720  &  1  &  0  &  3  &  2  &  1  &  0 \\ \hline
955  &  52  &  1  &  1  &  0  &  1867  &  525  &  1  &  1  &  0  &  {\color[HTML]{FE0000}5}  &  0  &  2 \\ \hline
1916  &  110  &  1  &  0  &  1  &  846  &  619  &  0  &  1  &  2  &  2  &  0  &  0 \\ \hline
604  &  106  &  1  &  0  &  1  &  2004  &  878  &  1  &  1  &  2  &  {\color[HTML]{FE0000}4}  &  1  &  0 \\ \hline
925  &  65  &  1  &  1  &  0  &  884  &  2036  &  1  &  1  &  3  &  1  &  1  &  2 \\ \hline
889  &  13  &  1  &  1  &  1  &  1599  &  1895  &  0  &  1  &  1  &  2  &  0  &  2 \\ \hline
1324  &  92  &  1  &  1  &  1  &  2009  &  692  &  0  &  1  &  2  &  1  &  0  &  1 \\ \hline
764  &  14  &  1  &  1  &  1  &  1076  &  1936  &  0  &  0  &  1  &  {\color[HTML]{FE0000}4}  &  1  &  2 \\ \hline
\end{tabular}
\end{table}
Población final:
\begin{table}[H]
\begin{tabular}{|l|l|l|l|l|l|l|l|l|l|l|l|l|}
\hline
\textbf{1} & \textbf{2} & \textbf{3} & \textbf{4} & \textbf{5} & \textbf{6} & \textbf{7} & \textbf{8} & \textbf{9} & \textbf{10} & \textbf{11} & \textbf{12} & \textbf{13} \\ \hline
1324  &  92  &  0  &  1  &  1  &  659  &  1251  &  0  &  1  &  0  &  {\color[HTML]{FE0000}5}  &  1  &  1 \\ \hline
604  &  106  &  1  &  0  &  1  &  2004  &  878  &  1  &  1  &  0  &  1  &  0  &  1 \\ \hline
666  &  17  &  0  &  0  &  1  &  846  &  619  &  0  &  1  &  2  &  2  &  0  &  0 \\ \hline
666  &  17  &  0  &  0  &  1  &  2009  &  692  &  0  &  1  &  0  &  2  &  0  &  0 \\ \hline
666  &  17  &  0  &  0  &  1  &  846  &  619  &  0  &  1  &  2  &  1  &  0  &  1 \\ \hline
666  &  17  &  0  &  1  &  1  &  2009  &  692  &  0  &  1  &  0  &  1  &  0  &  1 \\ \hline
666  &  17  &  0  &  1  &  1  &  659  &  1251  &  0  &  1  &  0  &  {\color[HTML]{FE0000}5}  &  1  &  1 \\ \hline
666  &  17  &  0  &  0  &  1  &  846  &  619  &  0  &  1  &  2  &  2  &  0  &  0 \\ \hline
604  &  106  &  1  &  0  &  1  &  2004  &  878  &  1  &  1  &  2  &  {\color[HTML]{FE0000}4}  &  1  &  0 \\ \hline
1324  &  92  &  1  &  1  &  1  &  2009  &  692  &  0  &  1  &  2  &  1  &  0  &  1 \\ \hline
\end{tabular}
\end{table}

\subsection{Población de 10 individuos durante 5 generaciones}
Resultados para una población de 10 individuos durante 5 generaciones:
Población inicial:
\begin{table}[H]
\begin{tabular}{|l|l|l|l|l|l|l|l|l|l|l|l|l|}
\hline
\textbf{1} & \textbf{2} & \textbf{3} & \textbf{4} & \textbf{5} & \textbf{6} & \textbf{7} & \textbf{8} & \textbf{9} & \textbf{10} & \textbf{11} & \textbf{12} & \textbf{13} \\ \hline
808  &  90  &  1  &  0  &  1  &  1675  &  1228  &  0  &  0  &  0  &  3  &  1  &  0 \\ \hline
1130  &  102  &  0  &  1  &  1  &  1347  &  1232  &  1  &  0  &  3  &  2  &  1  &  1 \\ \hline
886  &  81  &  0  &  0  &  1  &  1860  &  1851  &  0  &  1  &  3  &  {\color[HTML]{FE0000}5}  &  1  &  1 \\ \hline
1845  &  92  &  1  &  1  &  1  &  1978  &  1661  &  1  &  0  &  3  &  0  &  0  &  1 \\ \hline
1394  &  44  &  1  &  1  &  1  &  1986  &  975  &  0  &  1  &  0  &  3  &  0  &  1 \\ \hline
1444  &  73  &  1  &  1  &  1  &  1364  &  1532  &  0  &  1  &  0  &  {\color[HTML]{FE0000}5}  &  1  &  1 \\ \hline
960  &  22  &  1  &  0  &  0  &  780  &  1749  &  1  &  0  &  3  &  3  &  0  &  0 \\ \hline
1663  &  12  &  1  &  1  &  0  &  538  &  1004  &  0  &  1  &  3  &  2  &  0  &  0 \\ \hline
877  &  18  &  1  &  0  &  1  &  972  &  1588  &  1  &  0  &  1  &  1  &  0  &  2 \\ \hline
852  &  23  &  1  &  1  &  0  &  1881  &  1253  &  1  &  0  &  0  &  2  &  1  &  1 \\ \hline
\end{tabular}
\end{table}
Población final:
\begin{table}[H]
\begin{tabular}{|l|l|l|l|l|l|l|l|l|l|l|l|l|}
\hline
\textbf{1} & \textbf{2} & \textbf{3} & \textbf{4} & \textbf{5} & \textbf{6} & \textbf{7} & \textbf{8} & \textbf{9} & \textbf{10} & \textbf{11} & \textbf{12} & \textbf{13} \\ \hline
808  &  90  &  1  &  1  &  0  &  1675  &  1228  &  0  &  0  &  0  &  3  &  1  &  1 \\ \hline
808  &  90  &  1  &  1  &  0  &  1675  &  1228  &  0  &  0  &  0  &  3  &  1  &  0 \\ \hline
808  &  23  &  1  &  0  &  1  &  1675  &  1228  &  0  &  0  &  0  &  3  &  0  &  1 \\ \hline
808  &  23  &  1  &  0  &  0  &  1675  &  1228  &  0  &  0  &  0  &  3  &  0  &  1 \\ \hline
808  &  90  &  1  &  0  &  0  &  1675  &  1228  &  0  &  0  &  0  &  3  &  1  &  0 \\ \hline
808  &  23  &  1  &  0  &  0  &  1675  &  1228  &  0  &  0  &  0  &  3  &  1  &  0 \\ \hline
852  &  23  &  1  &  0  &  1  &  1675  &  1228  &  0  &  0  &  0  &  3  &  1  &  0 \\ \hline
852  &  23  &  1  &  0  &  1  &  1675  &  1228  &  0  &  0  &  0  &  3  &  0  &  1 \\ \hline
808  &  90  &  1  &  1  &  0  &  1675  &  1228  &  0  &  0  &  0  &  3  &  1  &  1 \\ \hline
808  &  90  &  1  &  1  &  0  &  1675  &  1228  &  0  &  0  &  0  &  3  &  1  &  1 \\ \hline
\end{tabular}
\end{table}

\subsection{Población de 20 individuos durante 20 generaciones}
Resultados para una población de 20 individuos durante 20 generaciones:
Población inicial:
\begin{table}[H]
\begin{tabular}{|l|l|l|l|l|l|l|l|l|l|l|l|l|}
\hline
\textbf{1} & \textbf{2} & \textbf{3} & \textbf{4} & \textbf{5} & \textbf{6} & \textbf{7} & \textbf{8} & \textbf{9} & \textbf{10} & \textbf{11} & \textbf{12} & \textbf{13} \\ \hline
1243  &  104  &  0  &  0  &  1  &  1308  &  1083  &  1  &  1  &  0  &  1  &  0  &  1 \\ \hline
821  &  90  &  0  &  1  &  0  &  1669  &  1685  &  1  &  0  &  3  &  {\color[HTML]{FE0000}4}  &  0  &  1 \\ \hline
729  &  31  &  0  &  1  &  1  &  1548  &  1278  &  1  &  0  &  1  &  1  &  0  &  2 \\ \hline
1265  &  11  &  1  &  1  &  0  &  1906  &  1049  &  1  &  1  &  2  &  {\color[HTML]{FE0000}4}  &  0  &  2 \\ \hline
831  &  10  &  1  &  1  &  0  &  1095  &  917  &  0  &  1  &  0  &  2  &  1  &  0 \\ \hline
1174  &  11  &  1  &  1  &  1  &  1049  &  1728  &  0  &  0  &  1  &  1  &  1  &  0 \\ \hline
832  &  56  &  0  &  0  &  0  &  1942  &  1548  &  1  &  0  &  3  &  2  &  1  &  1 \\ \hline
935  &  100  &  0  &  0  &  0  &  1562  &  1395  &  0  &  0  &  1  &  0  &  1  &  1 \\ \hline
1886  &  64  &  1  &  1  &  1  &  1533  &  1351  &  0  &  0  &  0  &  0  &  0  &  2 \\ \hline
724  &  70  &  0  &  0  &  1  &  831  &  1710  &  0  &  1  &  2  &  3  &  1  &  1 \\ \hline
519  &  44  &  1  &  0  &  1  &  1340  &  1555  &  1  &  0  &  2  &  0  &  1  &  1 \\ \hline
923  &  85  &  0  &  1  &  0  &  1910  &  1432  &  0  &  0  &  1  &  2  &  0  &  0 \\ \hline
667  &  60  &  1  &  0  &  1  &  1692  &  1686  &  1  &  1  &  0  &  0  &  1  &  0 \\ \hline
1959  &  15  &  0  &  1  &  1  &  1733  &  1242  &  1  &  1  &  3  &  0  &  0  &  2 \\ \hline
765  &  32  &  0  &  0  &  0  &  1442  &  1625  &  1  &  1  &  1  &  1  &  0  &  2 \\ \hline
1482  &  117  &  1  &  0  &  1  &  595  &  1477  &  1  &  0  &  0  &  {\color[HTML]{FE0000}5}  &  1  &  1 \\ \hline
1091  &  35  &  0  &  1  &  0  &  1389  &  2043  &  1  &  0  &  2  &  3  &  1  &  2 \\ \hline
524  &  79  &  0  &  1  &  0  &  1019  &  1730  &  0  &  0  &  3  &  {\color[HTML]{FE0000}5}  &  0  &  1 \\ \hline
1499  &  10  &  0  &  0  &  1  &  1042  &  1140  &  1  &  1  &  3  &  1  &  1  &  2 \\ \hline
966  &  12  &  1  &  0  &  0  &  1651  &  632  &  0  &  1  &  2  &  {\color[HTML]{FE0000}4}  &  0  &  0 \\ \hline
\end{tabular}
\end{table}
Población final:
\begin{table}[H]
\begin{tabular}{|l|l|l|l|l|l|l|l|l|l|l|l|l|}
\hline
\textbf{1} & \textbf{2} & \textbf{3} & \textbf{4} & \textbf{5} & \textbf{6} & \textbf{7} & \textbf{8} & \textbf{9} & \textbf{10} & \textbf{11} & \textbf{12} & \textbf{13} \\ \hline
524  &  18  &  0  &  0  &  0  &  1340  &  1686  &  0  &  1  &  0  &  0  &  1  &  0 \\ \hline
865  &  18  &  0  &  0  &  0  &  1340  &  1686  &  1  &  1  &  0  &  0  &  1  &  0 \\ \hline
524  &  39  &  0  &  0  &  0  &  1340  &  552  &  1  &  1  &  0  &  0  &  1  &  0 \\ \hline
524  &  39  &  0  &  0  &  0  &  1340  &  1686  &  0  &  1  &  0  &  2  &  1  &  0 \\ \hline
524  &  39  &  0  &  0  &  0  &  1340  &  1686  &  0  &  1  &  0  &  0  &  0  &  0 \\ \hline
524  &  40  &  0  &  0  &  0  &  1340  &  1686  &  1  &  1  &  0  &  0  &  1  &  0 \\ \hline
524  &  39  &  0  &  0  &  0  &  1340  &  1686  &  0  &  1  &  3  &  0  &  1  &  0 \\ \hline
524  &  34  &  0  &  0  &  0  &  1340  &  1686  &  0  &  1  &  0  &  0  &  1  &  0 \\ \hline
524  &  18  &  0  &  0  &  0  &  1340  &  1686  &  0  &  1  &  0  &  0  &  1  &  0 \\ \hline
806  &  34  &  0  &  0  &  0  &  1340  &  1686  &  0  &  1  &  0  &  0  &  1  &  0 \\ \hline
524  &  39  &  0  &  0  &  0  &  1340  &  1686  &  1  &  1  &  0  &  0  &  1  &  0 \\ \hline
524  &  18  &  0  &  0  &  0  &  1340  &  1952  &  0  &  1  &  0  &  0  &  1  &  0 \\ \hline
524  &  34  &  0  &  0  &  0  &  1340  &  1686  &  1  &  1  &  0  &  0  &  1  &  0 \\ \hline
767  &  39  &  0  &  0  &  0  &  1340  &  1686  &  1  &  1  &  0  &  0  &  1  &  0 \\ \hline
524  &  18  &  0  &  0  &  0  &  1340  &  1686  &  1  &  1  &  0  &  0  &  1  &  0 \\ \hline
524  &  39  &  0  &  0  &  0  &  1340  &  1686  &  1  &  1  &  0  &  0  &  1  &  0 \\ \hline
524  &  39  &  0  &  0  &  0  &  1340  &  1686  &  0  &  1  &  0  &  0  &  1  &  0 \\ \hline
524  &  39  &  0  &  0  &  0  &  1340  &  552  &  1  &  1  &  0  &  0  &  1  &  0 \\ \hline
524  &  34  &  0  &  0  &  0  &  1340  &  1686  &  1  &  1  &  0  &  0  &  1  &  0 \\ \hline
524  &  18  &  0  &  0  &  0  &  1340  &  1686  &  0  &  1  &  0  &  0  &  1  &  0 \\ \hline
\end{tabular}
\end{table}

\section{Cruzamiento en dos puntos}

\subsection{Población de 10 individuos durante 2 generaciones}
Resultados para una población de 10 individuos durante 2 generaciones:
Población inicial:
\begin{table}[H]
\begin{tabular}{|l|l|l|l|l|l|l|l|l|l|l|l|l|}
\hline
\textbf{1} & \textbf{2} & \textbf{3} & \textbf{4} & \textbf{5} & \textbf{6} & \textbf{7} & \textbf{8} & \textbf{9} & \textbf{10} & \textbf{11} & \textbf{12} & \textbf{13} \\ \hline
1997  &  107  &  0  &  1  &  0  &  1571  &  608  &  1  &  1  &  0  &  1  &  0  &  0 \\ \hline
1879  &  23  &  1  &  1  &  0  &  1378  &  1842  &  0  &  1  &  0  &  2  &  0  &  0 \\ \hline
1036  &  79  &  1  &  1  &  1  &  1687  &  763  &  1  &  0  &  3  &  {\color[HTML]{FE0000}4}  &  1  &  2 \\ \hline
1991  &  20  &  1  &  0  &  0  &  747  &  1662  &  1  &  1  &  1  &  0  &  0  &  2 \\ \hline
1793  &  43  &  0  &  0  &  0  &  1080  &  1822  &  1  &  0  &  0  &  0  &  0  &  0 \\ \hline
1471  &  81  &  0  &  1  &  0  &  1547  &  1390  &  1  &  0  &  2  &  {\color[HTML]{FE0000}5}  &  0  &  2 \\ \hline
1923  &  36  &  1  &  0  &  1  &  1269  &  717  &  0  &  0  &  1  &  {\color[HTML]{FE0000}4}  &  0  &  1 \\ \hline
961  &  47  &  1  &  0  &  1  &  634  &  1765  &  0  &  0  &  3  &  {\color[HTML]{FE0000}4}  &  0  &  0 \\ \hline
899  &  30  &  0  &  0  &  1  &  992  &  1202  &  0  &  1  &  1  &  2  &  0  &  2 \\ \hline
1423  &  78  &  0  &  1  &  0  &  1615  &  2027  &  0  &  0  &  0  &  {\color[HTML]{FE0000}5}  &  1  &  0 \\ \hline
\end{tabular}
\end{table}
Población final:
\begin{table}[H]
\begin{tabular}{|l|l|l|l|l|l|l|l|l|l|l|l|l|}
\hline
\textbf{1} & \textbf{2} & \textbf{3} & \textbf{4} & \textbf{5} & \textbf{6} & \textbf{7} & \textbf{8} & \textbf{9} & \textbf{10} & \textbf{11} & \textbf{12} & \textbf{13} \\ \hline
899  &  34  &  0  &  0  &  1  &  634  &  1765  &  0  &  0  &  0  &  1  &  0  &  1 \\ \hline
961  &  47  &  1  &  0  &  1  &  634  &  1765  &  0  &  0  &  3  &  {\color[HTML]{FE0000}4}  &  0  &  0 \\ \hline
961  &  47  &  1  &  0  &  1  &  634  &  1765  &  0  &  0  &  1  &  2  &  0  &  0 \\ \hline
899  &  47  &  1  &  0  &  1  &  634  &  1765  &  0  &  1  &  0  &  1  &  0  &  1 \\ \hline
961  &  47  &  1  &  0  &  1  &  634  &  1765  &  0  &  0  &  2  &  {\color[HTML]{FE0000}4}  &  0  &  0 \\ \hline
961  &  47  &  1  &  0  &  1  &  634  &  1765  &  0  &  0  &  3  &  {\color[HTML]{FE0000}4}  &  0  &  0 \\ \hline
961  &  47  &  1  &  0  &  1  &  634  &  1765  &  0  &  0  &  3  &  {\color[HTML]{FE0000}4}  &  0  &  0 \\ \hline
961  &  47  &  1  &  0  &  1  &  634  &  1765  &  0  &  0  &  3  &  {\color[HTML]{FE0000}4}  &  0  &  0 \\ \hline
899  &  30  &  0  &  0  &  1  &  992  &  1202  &  0  &  1  &  1  &  2  &  0  &  2 \\ \hline
899  &  30  &  0  &  0  &  1  &  992  &  1202  &  0  &  1  &  0  &  1  &  0  &  1 \\ \hline
\end{tabular}
\end{table}

\subsection{Población de 10 individuos durante 5 generaciones}
Resultados para una población de 10 individuos durante 5 generaciones:
Población inicial:
\begin{table}[H]
\begin{tabular}{|l|l|l|l|l|l|l|l|l|l|l|l|l|}
\hline
\textbf{1} & \textbf{2} & \textbf{3} & \textbf{4} & \textbf{5} & \textbf{6} & \textbf{7} & \textbf{8} & \textbf{9} & \textbf{10} & \textbf{11} & \textbf{12} & \textbf{13} \\ \hline
2008  &  57  &  1  &  1  &  0  &  1008  &  1736  &  1  &  0  &  2  &  3  &  0  &  0 \\ \hline
2036  &  115  &  1  &  1  &  0  &  715  &  1646  &  0  &  0  &  0  &  3  &  0  &  2 \\ \hline
806  &  10  &  0  &  0  &  1  &  763  &  1061  &  0  &  1  &  0  &  0  &  1  &  2 \\ \hline
1184  &  19  &  0  &  0  &  0  &  687  &  1944  &  0  &  0  &  0  &  3  &  1  &  2 \\ \hline
1244  &  100  &  0  &  0  &  1  &  1930  &  1088  &  0  &  0  &  0  &  1  &  0  &  2 \\ \hline
1498  &  46  &  0  &  1  &  0  &  819  &  1391  &  0  &  0  &  1  &  1  &  0  &  2 \\ \hline
2037  &  120  &  1  &  1  &  1  &  1137  &  1059  &  1  &  0  &  2  &  2  &  0  &  1 \\ \hline
1171  &  118  &  0  &  0  &  1  &  1564  &  569  &  0  &  1  &  1  &  {\color[HTML]{FE0000}4}  &  1  &  2 \\ \hline
714  &  120  &  1  &  0  &  0  &  1282  &  1579  &  0  &  1  &  2  &  1  &  1  &  2 \\ \hline
1710  &  87  &  0  &  0  &  0  &  654  &  529  &  0  &  0  &  2  &  1  &  1  &  2 \\ \hline
\end{tabular}
\end{table}
Población final:
\begin{table}[H]
\begin{tabular}{|l|l|l|l|l|l|l|l|l|l|l|l|l|}
\hline
\textbf{1} & \textbf{2} & \textbf{3} & \textbf{4} & \textbf{5} & \textbf{6} & \textbf{7} & \textbf{8} & \textbf{9} & \textbf{10} & \textbf{11} & \textbf{12} & \textbf{13} \\ \hline
714  &  120  &  0  &  0  &  0  &  687  &  1944  &  1  &  1  &  2  &  1  &  1  &  2 \\ \hline
714  &  43  &  0  &  0  &  0  &  1282  &  1944  &  0  &  0  &  2  &  1  &  1  &  0 \\ \hline
714  &  43  &  0  &  0  &  0  &  1282  &  1944  &  1  &  0  &  2  &  1  &  1  &  2 \\ \hline
714  &  120  &  0  &  0  &  0  &  1282  &  1579  &  0  &  1  &  1  &  1  &  1  &  0 \\ \hline
714  &  120  &  0  &  0  &  0  &  687  &  1579  &  1  &  1  &  2  &  1  &  1  &  2 \\ \hline
714  &  120  &  0  &  0  &  0  &  1282  &  1579  &  0  &  1  &  2  &  1  &  1  &  0 \\ \hline
714  &  120  &  0  &  0  &  0  &  687  &  1944  &  1  &  1  &  2  &  1  &  1  &  2 \\ \hline
714  &  43  &  0  &  0  &  0  &  1282  &  1579  &  0  &  0  &  2  &  1  &  1  &  2 \\ \hline
714  &  120  &  1  &  0  &  1  &  687  &  1944  &  1  &  0  &  2  &  1  &  1  &  2 \\ \hline
714  &  120  &  0  &  0  &  0  &  1282  &  1579  &  0  &  1  &  2  &  1  &  1  &  2 \\ \hline
\end{tabular}
\end{table}

\subsection{Población de 20 individuos durante 20 generaciones}
Resultados para una población de 20 individuos durante 20 generaciones:
Población inicial:
\begin{table}[H]
\begin{tabular}{|l|l|l|l|l|l|l|l|l|l|l|l|l|}
\hline
\textbf{1} & \textbf{2} & \textbf{3} & \textbf{4} & \textbf{5} & \textbf{6} & \textbf{7} & \textbf{8} & \textbf{9} & \textbf{10} & \textbf{11} & \textbf{12} & \textbf{13} \\ \hline
1677  &  14  &  0  &  1  &  0  &  804  &  1001  &  1  &  0  &  0  &  0  &  0  &  0 \\ \hline
1591  &  95  &  1  &  1  &  0  &  550  &  1731  &  0  &  0  &  1  &  0  &  0  &  2 \\ \hline
776  &  110  &  1  &  0  &  1  &  1205  &  1182  &  1  &  0  &  3  &  3  &  1  &  0 \\ \hline
1200  &  71  &  1  &  1  &  1  &  1903  &  1979  &  1  &  1  &  1  &  2  &  0  &  1 \\ \hline
1961  &  28  &  0  &  0  &  1  &  1211  &  1485  &  1  &  1  &  0  &  {\color[HTML]{FE0000}4}  &  0  &  2 \\ \hline
942  &  10  &  0  &  1  &  1  &  1446  &  1525  &  1  &  0  &  0  &  1  &  0  &  1 \\ \hline
537  &  48  &  1  &  0  &  1  &  1358  &  1930  &  1  &  0  &  3  &  1  &  1  &  1 \\ \hline
1105  &  10  &  1  &  0  &  0  &  571  &  1297  &  0  &  1  &  0  &  3  &  1  &  2 \\ \hline
1331  &  84  &  1  &  0  &  0  &  748  &  1126  &  0  &  0  &  2  &  {\color[HTML]{FE0000}5}  &  1  &  2 \\ \hline
2017  &  112  &  1  &  1  &  0  &  910  &  737  &  1  &  1  &  3  &  2  &  1  &  1 \\ \hline
770  &  96  &  0  &  0  &  1  &  1257  &  1804  &  1  &  1  &  1  &  3  &  0  &  1 \\ \hline
1569  &  104  &  1  &  1  &  1  &  713  &  671  &  0  &  0  &  0  &  3  &  0  &  1 \\ \hline
1280  &  120  &  1  &  1  &  1  &  1461  &  1404  &  1  &  1  &  0  &  1  &  0  &  1 \\ \hline
898  &  92  &  1  &  0  &  0  &  1943  &  1409  &  1  &  1  &  0  &  1  &  1  &  1 \\ \hline
1632  &  70  &  1  &  0  &  0  &  1821  &  1836  &  1  &  0  &  1  &  {\color[HTML]{FE0000}5}  &  0  &  1 \\ \hline
982  &  36  &  0  &  1  &  1  &  1462  &  1446  &  1  &  0  &  1  &  1  &  1  &  2 \\ \hline
1843  &  118  &  0  &  0  &  1  &  697  &  828  &  1  &  0  &  1  &  {\color[HTML]{FE0000}4}  &  0  &  0 \\ \hline
782  &  109  &  1  &  0  &  1  &  524  &  1928  &  1  &  1  &  1  &  {\color[HTML]{FE0000}4}  &  1  &  2 \\ \hline
1779  &  10  &  1  &  0  &  0  &  1627  &  1315  &  1  &  1  &  1  &  1  &  0  &  2 \\ \hline
1927  &  21  &  1  &  0  &  0  &  771  &  576  &  1  &  1  &  2  &  2  &  1  &  0 \\ \hline
\end{tabular}
\end{table}
Población final:
\begin{table}[H]
\begin{tabular}{|l|l|l|l|l|l|l|l|l|l|l|l|l|}
\hline
\textbf{1} & \textbf{2} & \textbf{3} & \textbf{4} & \textbf{5} & \textbf{6} & \textbf{7} & \textbf{8} & \textbf{9} & \textbf{10} & \textbf{11} & \textbf{12} & \textbf{13} \\ \hline
537  &  10  &  0  &  0  &  0  &  571  &  994  &  1  &  0  &  0  &  3  &  0  &  0 \\ \hline
537  &  10  &  0  &  0  &  0  &  571  &  994  &  0  &  0  &  0  &  3  &  0  &  0 \\ \hline
537  &  10  &  0  &  0  &  0  &  571  &  849  &  0  &  0  &  0  &  3  &  0  &  0 \\ \hline
537  &  10  &  0  &  0  &  0  &  571  &  994  &  0  &  0  &  0  &  3  &  0  &  0 \\ \hline
537  &  10  &  0  &  0  &  0  &  571  &  994  &  0  &  0  &  0  &  3  &  0  &  0 \\ \hline
537  &  10  &  0  &  0  &  0  &  571  &  994  &  0  &  0  &  1  &  3  &  0  &  0 \\ \hline
537  &  10  &  0  &  0  &  0  &  571  &  849  &  0  &  0  &  0  &  3  &  0  &  0 \\ \hline
537  &  10  &  0  &  0  &  0  &  571  &  849  &  0  &  0  &  0  &  3  &  0  &  0 \\ \hline
537  &  10  &  0  &  0  &  0  &  571  &  1285  &  0  &  0  &  0  &  3  &  0  &  0 \\ \hline
537  &  10  &  0  &  0  &  0  &  571  &  849  &  0  &  0  &  0  &  3  &  0  &  0 \\ \hline
537  &  10  &  0  &  0  &  0  &  1430  &  849  &  0  &  0  &  0  &  3  &  0  &  0 \\ \hline
537  &  10  &  0  &  0  &  1  &  571  &  994  &  0  &  0  &  0  &  3  &  0  &  0 \\ \hline
537  &  10  &  0  &  0  &  0  &  1028  &  994  &  0  &  0  &  0  &  3  &  0  &  0 \\ \hline
537  &  10  &  0  &  0  &  0  &  571  &  994  &  0  &  0  &  0  &  3  &  0  &  0 \\ \hline
537  &  10  &  0  &  0  &  0  &  571  &  994  &  0  &  0  &  0  &  3  &  0  &  0 \\ \hline
537  &  10  &  0  &  0  &  0  &  571  &  994  &  0  &  0  &  0  &  3  &  0  &  0 \\ \hline
537  &  10  &  0  &  0  &  0  &  571  &  994  &  0  &  0  &  0  &  3  &  0  &  0 \\ \hline
537  &  10  &  0  &  0  &  0  &  571  &  994  &  0  &  0  &  0  &  3  &  0  &  0 \\ \hline
537  &  10  &  0  &  0  &  0  &  571  &  994  &  0  &  0  &  0  &  3  &  0  &  0 \\ \hline
537  &  10  &  0  &  0  &  0  &  571  &  849  &  0  &  0  &  0  &  3  &  0  &  0 \\ \hline
\end{tabular}
\end{table}

\section{Resultados para 30 ejecuciones del algoritmo}

Tras la ejecución y consid

\section{Comentarios sobre los resultados}

Como podemos observar, en las generaciones iniciales hay configuraciones correctas e incorrectas, las cuales podemos identificar porque tienen algún valor en rojo. 

\bigskips
Con tan solo 2 generaciones y con una población de 10 individuos generados aleatoriamente la población final tiene configuraciones incorrectas pero aun así el algoritmo al ordenar la población en base al fitness nos posicionará al final de la lista la que considera la configuración más segura de toda la población. Aun así se aprecia que con tan pocas generaciones el algoritmo no es capaz de evolucionar para descartar las configuraciones incorrectas, esta casuística se da por igual utilizando el cruzamiento por un punto y el de dos puntos.

\bigskip
Si ya pasamos a 5 generaciones manteniendo la población en 10 individuos generados aleatoriamente en ambos casos ya empieza a eliminar las configuraciones incorrectas, en ambos casos por igual aunque se aprecia que el cruce por dos puntos tiene la ventaja de proporcionar una mayor diversidad en la población.

\bigskip
Como prueba final hemos ejecutado el algoritmo durante 20 generaciones con una población de 20 individuos generados aleatoriamente vemos que, tanto con la función de cruce en un punto como con la que lo hace en dos puntos, el algoritmo descarta las configuraciones incorrectas pero en este caso, al ser tantas generaciones perdemos diversidad aunque ganamos en seguridad según el valor indicado por ZAP.

\bigskip
Tanto en 5 como en 20 generaciones el algoritmo genético ha evolucionado la población hasta llegar a un estado cercano al ideal en cuanto a seguridad.

\bigskip
Podemos afirmar que la funciona de cruzamiento en dos puntos proporciona un mejor resultado en menos generaciones, pero al aumentar el número de las mismas esta ventaja ya no se aprecia de forma tan evidente.

\bigskip
Por lo tanto, una vez decidido que la función de cruzamiento en dos puntos proporciona mejores resultaos que la que lo hace ne un único punto sólo deberíamos variar el número de generaciones dependiendo del grado de seguridad y/o diversidad que quisiéramos obtener, siendo a mas generaciones un resultado mas seguro pero menos diverso y siendo a menos generaciones un resultado mas diverso pero menos seguro.


